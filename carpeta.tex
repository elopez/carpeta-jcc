\documentclass[12pt, a4paper]{article}

\usepackage[T1]{fontenc}
\usepackage{polyglossia}
\setdefaultlanguage{spanish}
%\usepackage[utf8]{inputenc}
\usepackage{graphicx}
\usepackage{amsmath,amssymb,amsfonts,latexsym,cancel}
\usepackage{color}
\usepackage{rotating}
\usepackage{array}
\usepackage{multirow}
\usepackage{tikz}
\usepackage{lastpage}
\usepackage{eso-pic}
\usepackage{wrapfig}

%---------------------------------------------------------------------------------------------------
% Estilo de la página

\textheight = 26cm
\textwidth = 18.75cm
\topmargin = -1cm
\oddsidemargin = -1.25cm

\setlength{\parindent}{0cm}

\pagestyle{empty}

\AddToShipoutPicture{
	\begin{tikzpicture}
		\draw[white] (0,0) rectangle (0,0); % para setear el 0.

	% Logo LCC
		\path (18,28.5) node {\includegraphics[scale=0.1]{res/lambda-bn.pdf}};

	% JCC
		\path (5,28.75) node {\normalsize\bf XIII JORNADAS DE};
		\path (5,28.25) node {\normalsize\bf CIENCIAS DE LA COMPUTACIÓN};

	% Hoja ? de ?
		%\path (18,0.5) node {\footnotesize Hoja {\thepage} de \pageref*{LastPage}};

	% Lineas
		\draw[thick] (0.25,27.5) rectangle (20.5,27.5);
		%\draw[thick] (0.25,1) rectangle (20.5,1);

	\end{tikzpicture}
}

% Tipo de letra
%\renewcommand{\familydefault}{\sfdefault}
%---------------------------------------------------------------------------------------------------


%---------------------------------------------------------------------------------------------------
% Otros efectos
\everymath{\displaystyle}
%---------------------------------------------------------------------------------------------------


%---------------------------------------------------------------------------------------------------
% Comandos y macros

% Colores
\definecolor{colorJCC}{rgb}{1, 0.4, 0}


% Espacios truchos
\def\ne{\vspace{-12pt}} % negative enter
\def\se{$ $\\}		% simple enter
\def\de{\se\newline}	% doble enter
\def\te{\de\newline}	% triple enter


% Separar en sílabas
%\hyphenation{in-te-rrup-ci\'on}


% Dibujos
\newenvironment{dibujo}{\begin{center}\begin{tikzpicture}}{\end{tikzpicture}\end{center}}
\newenvironment{dibujoNC}{\begin{tikzpicture}}{\end{tikzpicture}}

% Tablero
\newcommand{\tablero}[1]{
	\pgfmathsetmacro{\boardsize}{#1}

	\foreach \i in {0,...,\boardsize}{
		\foreach \j in {0,...,\boardsize}{
			\fill[bola] (\i,\j) circle(2.3pt);
		}
	}
}

\newcommand{\tableroraya}[1]{
	\pgfmathsetmacro{\boardsize}{#1}
	\draw[step=1cm,gray,very thin] (0,0) grid (\boardsize,\boardsize);
	\draw[ultra thick,-] (0,0) rectangle (\boardsize,\boardsize);
}



% Descripcion charla
\newcommand{\charla}[3]{
	{\large\bf#1}\\
	{\bf por #2}\\
	#3\\
}
% \charla{Título}{Autor}{Descripción}

%---------------------------------------------------------------------------------------------------

\def\barra{\textbackslash}

\newcolumntype{C}[1]{>{\centering\let\newline\\\arraybackslash\hspace{0pt}}m{#1}}
\renewcommand{\arraystretch}{2.5}


%--------------------------------------------------------------------------------------------------------------------------------------------------------
%--------------------------------------------------------------------------------------------------------------------------------------------------------

\tikzset{
    bola/.style =  {fill=black, opacity=1},
    digito/.style = {font=\large\ttfamily},
}




\usepackage{tcolorbox}% http://ctan.org/pkg/tcolorbox
\definecolor{darkpad}{gray}{0.7}% Rule colour
\definecolor{lightpad}{gray}{1}% Rule colour
\makeatletter
\newcommand{\mousepad}[2]{%
  \begin{tcolorbox}[colframe=black,colback=#1,boxrule=0.5pt,arc=4pt,
      left=6pt,right=6pt,top=6pt,bottom=6pt,boxsep=0pt,width=8em,height=6em,nobeforeafter,halign=center,valign=center]
    \bf{#2}
  \end{tcolorbox}
}
\makeatother


\begin{document}
%\renewcommand{\arraystretch}{0.8}
\begin{center}{\LARGE\bf Cronograma de actividades}\\ %\end{center}
%\se
%\begin{center}
%\scalebox{0.85}{
%\begin{tabular}{|c|C{130pt}|C{130pt}|C{130pt}|} \hline
%\bf Hora $|$ Día & \large\bf Miércoles 15 & \large\bf Jueves 16 & \large\bf Viernes 17 \\ \hline
%
%\bf 9:00 - 12:00 & \multirow{2}{*}{} & Taller: Detectando corrupción de memoria con Ptrace en aplicaciones reales & Taller: Detectando corrupción de memoria con Ptrace en aplicaciones reales\\ \cline{1-1}\cline{3-4}
%
%\bf 12:00 - 14:00 & & \multicolumn{2}{c|}{\large\bf Almuerzo}\\[-5pt] \cline{1-1}\cline{3-4}
%
%\bf 14:00 - 14:45 &  & \multirow{2}{130pt}{\centering Métodos de navegación basada en visión para robots móviles} & \multirow{2}{130pt}{\centering Inteligencia Artificial y Gobierno Electrónico}\\ \cline{1-2}
%
%\bf 14:45 - 15:00 & \bf Acto de Apertura & & \\ \hline
%
%\bf 15:00 - 16:00 & Desarrollo de aplicaciones móviles multi-plataforma con Intel® XDK & IEPY: Una plataforma para Extracción de Información en Python & Análisis estático con clang-llvm\\ \hline
%
%\bf 16:00 - 17:00 & Innovación tecnológica en Neuralsoft & \multicolumn{2}{c|}{\large\bf Coffee Break}\\ \hline
%
%\bf 17:00 - 18:00 & \large\bf Coffee Break & Imaginando... & Lambda cálculo módulo isomorfismos de tipos\\ \hline
%
%\bf 18:00 - 19:00 & Ingeniería de Confiabilidad en Google & Recuperación de Información de Gran Escala & Certificación de algoritmos criptográficos constant-time\\ \hline
%
%\bf 19:00 - 20:00 & \begin{tabular}{C{118pt}}\small\textbf{¿Para qué estudiar Ciencias de la Computación?}\\[-12pt] Muestra de trabajos prácticos\end{tabular} & Bitcoin en profundidad: Un viaje criptográfico & \large\bf Entrega de diplomas\\ \hline
%
%\bf 22:45 - $\infty$ & \multicolumn{2}{c|}{} & \large\bf Cena de camaradería{\small $^{(*)}$}\\ \hline
%\end{tabular}
%}
%\end{center}
%\begin{center}
\hspace*{20pt}\includegraphics[scale=.85]{horario.pdf}
\end{center}

%---------------------------------------------------------------------------------------------------
\newpage

\charla
{Ciencias de la Computación en la escuela Argentina}
{Pablo Factorovich (Fundación Sadosky)}
{La iniciativa interministerial Program.AR ha conseguido importantes avances en pos de la inclusión de contenidos de ciencias de la computación dentro de los contenidos obligatorios de las escuelas argentinas. En esta charla comentaremos dichos progresos, los pasos que esta iniciativa encara para el futuro, así como la importancia de la comunidad académica de las Ciencias de la Computación en esta problemática.}

\charla
{Finding scalable solution for a worldwide audience}
{Sergio Giro (Google)}
{Because of worldwide massive impact, Google engineers face unique engineering problems every day. As a smart community, once they find a solution to an oft-recurring challenge they build a tool as to avoid reinventing the wheel every time. Tools are improved over time, and extended in order to fit new cases, leading to a mighty tool shed. In this talk I'll highlight some of these challenges and explain what our tools do to solve it, giving you a glance of how we work in this environment.}

\charla
{Innovación tecnológica en NeuralSoft}
{Gustavo Viceconti (NeuralSoft)}
{NeuralSoft, desde sus inicios, se ha caracterizado por la búsqueda de soluciones tecnológicas creativas e innovadoras que mejoren el modo de gestionar las empresas. \\
Al cabo de 26 años de trayectoria, la empresa rosarina se ha convertido en la principal fábrica de software ERP del país y es pionera y líder en Cloud Computing. Uno de sus principales logros ha sido acercar a las Pymes una tecnología de avanzada que hasta entonces sólo estaba disponible para grandes corporaciones. \\
NeuralSoft invierte casi el 20\% de su facturación en Investigación y Desarrollo e incorpora mejoras tecnológicas continuas a sus productos. \\
En permanente alianza con consultores y desarrolladores internacionales, hoy la empresa se prepara para la expansión global con el lanzamiento de una nueva plataforma tecnológica que tiene por objetivo generar un cambio en el paradigma clásico del software ERP para que el usuario pueda anticiparse a los posibles escenarios mediante técnicas de inteligencia artificial y simulación. Esto marcará un antes y un después en el modo de gestionar las empresas.}

\charla
{¿Para qué estudiar Ciencias de la Computación?}
{Martín Ceresa y Martín Escarrá (FCEIA - UNR)}
{Las ciencias informáticas han tomado gran relevancia en las últimas décadas,
convirtiéndose en el área más multidisciplinaria de hoy en día. ¿Cuáles son
entonces las subáreas más relevantes y qué estudia cada una? En esta charla
responderemos a estas preguntas y haremos foco en la Licenciatura en Ciencias
de la Computación para ayudar a armar el rompecabezas.}

\charla
{Programación Competitiva}
{Mariano Crosetti, Pablo Zimmermann y Martín Villagra (FCEIA - UNR)}
{La Programación Competitiva ha cobrado popularidad en los últimos años. Es una forma muy amena de profundizar en el uso y entendimiento de Estructuras de Datos y Algoritmos. El objetivo de la charla es iniciar a quien poco (o ningún) contacto ha tenido con el tema, y encauzar a quienes estén interesados a entrenar e incentivar la participación. Se introducirá el tipo de problemas que se trata y la metodología con la que se trabaja a la hora de entrenar algoritmia para competencias. Además los invitaremos a mantenerse en contacto con nosotros y el resto de la comunidad de Argentina y Latinoamérica para entrenar en conjunto.}

\charla
{Historia DesafiAR}
{Juan Manuel Baruffaldi y Santiago Iwakawa (DesafiAR)}
{DesafiAR es un juego de simulación de negocios desarrollado por alumnos de la carrera. Busca incentivar el espíritu emprendedor y capacitarlos en el camino. Se basa en gamification, es decir, que aprendan jugando.}

\charla
{Historia Disruptive SCience}
{Nicolás Pellejero, Juan Manuel Baruffaldi y Santiago Iwakawa (Disruptive SCience)}
{Emprendimiento que busca mejorar las relaciones de los clientes con las marcas por medio de la ciencia. Utiliza reconocimientos de rostros y de emociones por video. Historia de nuestro paso por la carrera y los trabajos hechos que llevaron al nacimiento de este proyecto.}

\charla
{Introducción a la computación cuántica}
{Alejandro Díaz-Caro (UNQ)}
{La computación cuántica es un paradigma de computación basado en la física cuántica. La idea es tomar la descripción de la evolución de sistemas físicos cuánticos como un proceso de cómputo: éstos tienen un estado inicial, un estado final, y la evolución del sistema puede ser descripta a través de ``operadores". Uno de los principales intereses en computación cuántica desde las ciencias de la computación es que existen algoritmos en este modelo de cómputo que tienen ganancias en complejidad con respecto a algoritmos clásicos, en algunos casos llegando hasta ganancias exponenciales. En este mini-curso de dos días se verá una introducción a la computación cuántica, así como algunos ejemplos de algoritmos y una aplicación a criptografía.}

\charla
{La Computación en Tiempos de Paralelismo}
{María Fabiana Piccoli (FCFMyN - UNSL)}
{El auge y la popularidad de nuevas tecnologías y la constante demanda de rápidos resultados en las aplicaciones hizo imprescindible el estudio de nuevas técnicas y metodologías de programación; la computación paralela es una de ellas.
Resolver un problema aplicando técnicas de computación paralela implica plantear una solución computacional donde varias unidades de procesamiento (hardware y software) trabajan cooperativamente y al mismo tiempo para obtener resultados más rápido que en la CPU, o en un tiempo razonable.
Varios aspectos deben ser considerados en una solución paralela, pudiendo existir varias o incluso ninguna, todo depende de la naturaleza del problema a resolver.
Hoy en día, los recursos computacionales existentes nos ofrecen un ambiente apto para el desarrollo de computación paralela. Esto no se limita a las arquitecturas multi-core, sino también a otros dispositivos distintos. Un ejemplo de ello son las placas de video o co-procesadores gráficos (GPUs). Ellos constituyen una arquitectura paralela, de bajo costo y alto rendimiento, capaz de resolver problemas de propósito general aplicando técnicas paralelas.
La presente charla tiene como objetivo exponer la necesidad de incluir técnicas paralelas en las soluciones computacionales de problemas de propósito general, mostrando que las GPUs constituyen una opción válida a la hora de pensar en arquitecturas paralelas de cómputo masivo. Finalmente, resultados de su utilización son detallados.}

\charla
{Análisis de resiliencia en sistemas embebidos}
{Pedro D'Argenio (FaMAF - UNC)}
{El estado actual de la tecnología demanda que los sistemas de computación brinden servicios en los que se pueda confiar justificadamente. En consecuencia, un sistema como éstos debe tolerar la ocurrencia inesperada de fallas, ataques o accidentes, mientras que su ejecución normal continúa respondiendo aceptablemente a parámetros de eficiencia, disponibilidad, confiabilidad, etc., y, obviamente, sin alterar su funcionamiento correcto. La mayoría de los eventos que comprometen el funcionamiento del sistema, al igual que muchas de las actividades en sus componentes, pueden cuantificarse probabilísticamente, lo cual permite hacer un análisis cuantitativo del sistema compuesto. En muchos casos, las propiedades a analizar requieren un alto grado de resiliencia. Una propiedad típica es solicitar 99.999\% de disponibilidad (``cinco nueves"). Lograr determinar propiedades cuya probabilidad de falla sea muy baja en sistemas complejos puede ser computacionalmente muy demandante. Para este tipo de problemas se idearon técnicas específicas para la simulación de eventos raros. Debido a su generalidad, nosotros hemos escogido enfocarnos en las denominadas técnicas por división de importancia. La efectividad de estas técnicas radica precisamente en encontrar buenas funciones de importancias que reflejen adecuadamente la cercanía del evento raro.
En esta charla explicaré las técnicas por división de importancia para simulación de eventos raros y nuestro trabajo actual en la derivación automática de funciones de importancia.
Completaré la charla con un breve resumen de nuestro grupo de investigación (http://dsg.famaf.unc.edu.ar) y de las actividades relacionadas a este proyecto específico de análisis de resiliencia.}

\charla
{Cotas Inferiores de Complejidad e Ingeniería de Software}
{Andrés Rojas Paredes (FCEyN - UBA)}
{Una cota inferior de complejidad permite afirmar que todos los algoritmos que resuelven un problema dado, tienen al menos una determinada complejidad. Ésto incluye a los algoritmos conocidos hoy, como también los algoritmos que podrían ser descubiertos en el futuro. Este tipo de resultado refleja un límite para la capacidad de hacer un software eficiente. \\
Sin embargo, obtener una cota inferior no es fácil y requiere definir tanto el modelo de cómputo sobre el que se definen los algoritmos como la medida de complejidad. La dificultad de este procedimiento es que los modelos de cómputo clásicos, como la máquina de Turing, son muy generales y no permiten resultados significativos para ciertos problemas, como por ejemplo, algoritmos de eliminación de variables. \\
En esta charla se presenta un modelo de cómputo que incorpora nociones de Ingeniería de Software. Esta característica permite demostrar cotas inferiores de complejidad exponenciales para los algoritmos actuales de eliminación de variables. En particular, nos referimos a algoritmos construidos bajo el paradigma orientado a objetos que implementan requerimientos no funcionales como la robustez. \\
Este modelo fue realizado en el marco de mi tesis de doctorado bajo la dirección del Doctor Joos Heintz.}

\charla
{Sobre enseñanza de Matemática y Programación en una carrera de Ingeniería de Software}
{Álvaro Tasistro (Universidad ORT - Uruguay)}
{La intención es presentar y discutir propuestas y experiencias acerca de cursos fundamentales de Matemática y Programación en el contexto de carreras de Ingeniería de Software, así como sus premisas iniciales, de las cuales las principales son:
\begin{itemize}
\item Que la Ingeniería de Software se beneficiaría de la incorporación de una teoría que habilitara a los practicantes a razonar metódica y explícitamente sobre los programas; es decir, una matemática de los programas, en contraposición al tratamiento de éstos como ``cajas negras". (Llamamos a esto ``Programación como Matemática".)
\item Que es interesante investigar el desarrollo de una Matemática cuyos objetos son datos y programas, siendo sus nociones básicas las de tipo de dato y algoritmo. (Llamamos a esto ``Matemática como Programación".)
\end{itemize}
Siendo que ninguna de esas dos ideas es enteramente novedosa, quizá lo sea su puesta en obra en el plano de la educación en cursos concretos basados en la Programación Funcional y la Matemática fundamentada en ella. Esta experiencia se pondrá a consideración en la charla.}

\charla
{Localización y Reparación Automática de Fallas}
{Marcelo Frias (ITBA)}
{En esta charla presentaré trabajo reciente sobre localización automática de fallas a partir de código con contratos, así como técnicas para la reparación automática de las mismas utilizando mutación de código. En la charla discutiré los fundamentos de las técnicas y presentaré demos de herramientas que las implementan. Presentaré también temas de trabajo dentro del grupo que pueden ser de interés para estudiantes que deseen continuar estudios de posgrado. \\
Trabajo conjunto con Luciano Zemín, Santiago Bermúdez, Nazareno Aguirre, Simón Gutierrez Brida y Santiago Perez de Rosso.}

\charla
{CIAA NXP: Alcances y limitaciones de un port basado en Buildroot/Linux}
{Ezequiel García (VanguardiaSur)}
{Hace ya varios años que el kernel Linux soporta arquitecturas sin MMU. En particular, el soporte para ARM cortex-M fue agregado oficialmente en 2013 y el soporte para NXP LPC43xx en 2015. \\
Esta charla se divide en tres partes. En primer lugar, haremos un repaso de los componentes que permiten correr Linux en la CIAA. \\
Luego, se analizarán las limitaciones que surgen de correr un sistema basado en Linux sobre una arquitectura sin MMU. \\
Finalmente, relevaremos el estado actual del soporte, tanto del kernel como del stack en espacio de usuario.}

\charla
{Ciencias de la Computación en la UNR: 20 años}
{Martín Degrati y Raúl Kantor (UNR)}
{El Dr. Raúl Kantor, primer Director de la carrera, junto al Lic. Martín Degrati, primer egresado de la Licenciatura en Cs. de la Computación en Rosario, comentarán sobre la historia de la carrera en Rosario.}

{\large\bf{Entrega de diplomas}}\\
{El Departamento de Ciencias de la Computación reconoce a aquellos alumnos que han completado el cursado de la Licenciatura en Ciencias de la Computación en el año 2015.}

%---------------------------------------------------------------------------------------------------
\newpage

\begin{center}\LARGE\bf Juegos\end{center}
%¿Estás aburrido? Te proponemos actividades que no molestan a los disertantes.\\

\section*{Que pase el que sigue}
Encontrá el elemento que sigue en las siguientes sucesiones:
\begin{itemize}
\item 2, 2, 4, 4, 2, 6, 6, 2, 8, 8, 16, 8, 24, 8, 32, $\dots$
\item 61, 23, 46, 821, 652, $\dots$
\item 2, 3, 5, 7, 1, 3, 7, 9, 6, $\dots$
\item 21, $e$, $\sqrt{-1}$, 93, 30, $\dots$
\item 2, 6, 6, 7, 8, 8, 2, 2, 4, 6, $\dots$
\item 1, 11, 21, 1211, 111221, 312211, $\dots$
\item 1888, 1892, 1896, 1904, 1908, $\dots$
\item 1, 3, 7, 12, 18, 26, 35, 45, 56, $\dots$
\end{itemize}


\section*{Robot buscaminas}

\begin{wrapfigure}{r}{0.34\textwidth}
\vspace{-2em}
\begin{dibujo}[scale=0.75]
\tablero{8}
\node[digito] at (0.5, 0.5) {3};
\node[digito] at (0.5, 2.5) {3};
\node[digito] at (0.5, 5.5) {3};
\node[digito] at (0.5, 7.5) {3};
\node[digito] at (2.5, 0.5) {3};
\node[digito] at (2.5, 2.5) {3};
\node[digito] at (2.5, 5.5) {3};
\node[digito] at (2.5, 7.5) {3};
\node[digito] at (5.5, 0.5) {3};
\node[digito] at (5.5, 2.5) {3};
\node[digito] at (5.5, 5.5) {3};
\node[digito] at (5.5, 7.5) {3};
\node[digito] at (7.5, 0.5) {3};
\node[digito] at (7.5, 2.5) {3};
\node[digito] at (7.5, 5.5) {3};
\node[digito] at (7.5, 7.5) {3};
\node[digito] at (3.5, 1.5) {2};
\node[digito] at (4.5, 1.5) {1};
\node[digito] at (3.5, 6.5) {0};
\node[digito] at (4.5, 6.5) {2};
\node[digito] at (1.5, 3.5) {1};
\node[digito] at (1.5, 4.5) {1};
\node[digito] at (6.5, 4.5) {0};
\node[digito] at (6.5, 3.5) {2};
\end{dibujo}
\vspace{-2.1em}
\end{wrapfigure}

El robot buscaminas es una versión del buscaminas muy jugada por los LCCs
de Washington. Fue creada por el señor Factura Puertas hace muchos años,
y, si bien los gráficos del juego dejan bastante que desear, es muy adictivo. \\

Para ganar hay que guiar a nuestro robot a través del campo minado,
trazando un circuito cerrado (es decir, un camino que
empiece y termine en el mismo punto). Para esto debemos dibujar
segmentos horizontales y verticales que conecten puntos adyacentes de
la cuadrícula. El circuito no puede pasar dos veces por un mismo punto, y
los números en las casillas indican cuántos de los cuatro lados que lo
rodean forman parte del circuito. ¿Te animás a pasar el nivel de la derecha?

\section*{Donde dice donde dice debe decir debe decir, y donde dice debe decir debe decir donde dice}

Ayer por la noche, Julio entró al castillo de Sir Casteador a recuperar su
auto-grafo original que Casteador le había robado. Sin embargo, se tropezó
y tiró la preciada colección de pads blancos y negros del Sir y se desordenaron.
¡Ayudalo a ordenarlos antes de que Sir Casteador lo descubra!

En total hay 8 pads con distintas inscripciones: la mitad son
blancos y el resto son negros. Julio sabe que estaban ordenados uno
atrás de otro, de manera que todas las frases resulten verdaderas.

\mbox{}\hfill
\mousepad{lightpad}{Los dos siguientes son negros}\hfill
\mousepad{lightpad}{Los dos siguientes son de distinto color}\hfill
\mousepad{lightpad}{El anterior es del mismo color que el siguiente}\hfill
\mousepad{lightpad}{Hay tantos pads negros antes como después}\hfill\mbox{}\\

\hfill
\mousepad{darkpad}{El anterior es del mismo color que el siguiente}\hfill
\mousepad{darkpad}{El anterior es blanco}\hfill
\mousepad{darkpad}{Los dos siguientes son del mismo color}\hfill
\mousepad{darkpad}{El anterior es negro}\hfill\mbox{}

\section*{El problema no es Sir Casteador}

\begin{wrapfigure}{r}{0.38\textwidth}
\vspace{-2.1em}
\begin{dibujo}[scale=0.85]
\tableroraya{8}
\draw[ultra thick,-] (3,0) -- (3,1) -- (1,1) -- (1,3) -- (0,3);
\draw[ultra thick,-] (3,8) -- (3,7) -- (1,7) -- (1,5) -- (0,5);
\draw[ultra thick,-] (5,0) -- (5,1) -- (7,1) -- (7,3) -- (8,3);
\draw[ultra thick,-] (5,8) -- (5,7) -- (7,7) -- (7,5) -- (8,5);

\draw[ultra thick,-] (1,3) -- (1,4) -- (2,4) -- (2,7);
\draw[ultra thick,-] (3,1) -- (3,6) -- (5,6) -- (5,7);
\draw[ultra thick,-] (4,6) -- (4,2) -- (6,2) -- (6,3) -- (5,3) -- (5,4) -- (7,4) -- (7,5);
\end{dibujo}
\vspace{-2.1em}
\end{wrapfigure}

El malvado computólogo Sir Casteador ha regresado para instaurar una
época de terror en la LCC, y esta vez es más peligroso que nunca. Cuentan
los antiguos \emph{diskettes} que para poder derrotar a Sir Casteador, uno
debe primero encontrar y destruir su colección de discos de Arjona,
que están escondidos en algunas casillas del tablero de la derecha. \\
Sabemos que en cada fila, en cada columna y en cada región remarcada hay
exactamente un disco, y que dos casillas que contengan un disco no pueden
ser vecinas (¡ni siquiera en diagonal!). \\
Ayúdennos a salvar al mundo de la amenaza de Arjona y Sir Casteador,
indicando en qué casillas están escondidos los discos. Tal vez debas
asistir al taller de programación competitiva para mejorar tus
habilidades si es que no puedes lograrlo rápidamente.

\section*{Tutorías LCC}
\begin{wrapfigure}{l}{0.36\textwidth}
\vspace{-1.35em}
\includegraphics[scale=0.4]{res/zoolander}
\end{wrapfigure}

Los tutores de este año consiguieron un robot programable para divertirse
e introducir a los tutorandos al mundo del software. Lamentablemente,
el robot vino un poco fallado y sólo puede avanzar o girar a la derecha.\\

Como desafío, los tutores le pidieron a los alumnos que encuentren un
camino de A a B en el laberinto de la izquierda, y que programen al robot
para que realice ese recorrido.\\

Los chicos están medio cansados y no quieren buscar el camino, ¿Podrías
ayudarlos y decirles por dónde debe pasar?

%---------------------------------------------------------------------------------------------------
%\newpage
%\begin{center}
%\includegraphics[scale=0.5,angle=90]{mygoto.png}
%\end{center}

\end{document}
